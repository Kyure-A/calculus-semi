\RequirePackage{plautopatch}
\RequirePackage[l2tabu, orthodox]{nag}

\documentclass[platex,dvipdfmx]{jlreq}			% for platex
% \documentclass[uplatex,dvipdfmx]{jlreq}		% for uplatex
\usepackage{graphicx}
\usepackage{bxtexlogo}
\usepackage{tcolorbox}
\usepackage{amsmath}
\usepackage{amssymb}
\usepackage{amsfonts}

\title{微積}

\author{kyre}
\date{\today}
\begin{document}
\section*{諸注意}
言い訳がましいが,俺は数学記号がエアプなのでイキって間違った記述をしている場合がある.説明と異なるような記述があれば,適宜指摘をお願いしたい.

\section{実数と数列}

\subsection{実数の連続性}

\subsubsection{上界と下界}
\begin{tcolorbox}
\begin{itemize}
\item 上界 

\begin{math}
\forall x \in S, x \leq a ならば, a を S の上界という.
\end{math}

\item 下界 

\begin{math}
\forall x \in S, a \leq x ならば,a を S の下界という.
\end{math}
\end{itemize}
\end{tcolorbox}
また,実数の部分集合 S が上界をもつとき,S を上に有界であるといい,S が下界をもつときは下に有界であるという.また,これら両方をもつとき S は有界であるという.(上/下界を "持つ" だとだめっぽい?)

\subsubsection{上限と下限}
\begin{tcolorbox}
\begin{itemize}
\item 上限 

\begin{math}
S \subset \mathbb{R}
\end{math}
\item 下限
\end{itemize}
\end{tcolorbox}

\subsubsection{実数の連続性}
























\end{document}